\documentclass{article}

% Language setting
% Replace `english' with e.g. `spanish' to change the document language
\usepackage[french]{babel}

% Display roman numbers instead or arabic numbers in titles of sections and subsections
\renewcommand{\thesection}{\Roman{section}}
\renewcommand{\thesubsection}{\thesection.\Roman{subsection}}

% Set page size and margins
% Replace `letterpaper' with `a4paper' for UK/EU standard size
\usepackage[letterpaper,top=2cm,bottom=2cm,left=3cm,right=3cm,marginparwidth=1.75cm]{geometry}

% Useful packages
\usepackage{amsmath}
\usepackage{graphicx}
\usepackage[colorlinks=true, allcolors=blue]{hyperref}

\title{Compte-rendu projet informatique : Simulation à N corps et système solaire}
\author{COUËRON Lola \\ JOLY Marine \\ \\ L3 Physique, Université Grenoble Alpes}

\begin{document}
\maketitle

{
\hypersetup{hidelinks}

\renewcommand{\contentsname}{Sommaire}
\tableofcontents
}

\newpage

\section{Introduction}
    \subsection{Contexte}
    La simulation informatique est un outil de plus en plus utilisé dans l'étude des phénomènes physique puisqu'elle permet de comparer de façon efficace des modèles théoriques aux résultats expérimentaux. Celle-ci permet également de simuler des expériences qui ne sonr pas réalisables en pratique. Dans ce cadre, nous avons développé une simulation à N corps en Python. En partant de la modélisation de l'interaction Terre-Soleil avec les équations de Newton puis d'une simulation du système solaire à l'aide de l'algorithme de Runge-Kutta d'ordre 4, nous avons créé un logiciel qui permet de personnaliser des simulations avec autant de corps que souhaité.

    \subsection{Structure générale du code}
    Ce projet étant collaboratif, nous avons travaillé sur \href{https://github.com/MJ240103/solarSystem}{GitHub} afin d'optimiser la répartition des tâches. Ce projet est divisé en 4 fichiers types de fichiers : \\

    \begin{enumerate}
        \item le fichier contenant l'application réalisée avec Tkinter (UIsolarSystem.py)
        \item le fichier contenant la simulation réalisée avec PyGame (mainEngine.py)
        \item le fichier contenant les équations et les paramètres physiques nécessaires à la simulation (bodies.py)
        \item les fichier permettant de stocker les paramètres de simulations déjà existantes (fichiers json d'extension .mj et Simulation1.py)
    \end{enumerate}

    \\

    \begin{figure}[h]
        \centering
        \includegraphics[width=0.5\linewidth]{imgGitHub.png}
        \caption{\label{fig:Git}Fichiers de la branche main du repository solarSystem.}
    \end{figure}

    \\
    
    Le fichier UIsolarSystem utilise

\section{Partie physique}
    \subsection{Implémentation naïve}

    \subsection{Implémentation avec Runge-Kutta d'ordre 4}

\section{Partie interface}
    \subsection{Interface de simulation avec pygame}

    \subsection{Intrface de l'application avec Tkinter}

\section{Exemples d'application}
    \subsection{Cas à deux corps}

    \subsection{Cas du système solaire}

\section{Conclusion et ouverture}


\end{document}